\section{OMR Architecture}

In general, the challenge of OMR can be decomposed into more defined sections. Although implementations obviously differ, an overview of some of the techniques I have come across for the various `common' stages which I researched can be summarised by:

\begin{table}[h]
\footnotesize
\begin{tabularx}{\linewidth}{ X X X X }
\toprule
  \textbf{1. Pre Processing} & \textbf{2. Segmentation} & \textbf{3. Classification} &  \textbf{4. Reconstruction} \\
  \midrule

  \begin{enumerate}[leftmargin=*]
    \item Level Adjustment
    \item Binarization
    \item Noise Removal
    \item Handling Skew
    \item Rotation
    \item Stave Removal
      \begin{enumerate}
        \item Horizontal Projections
        \item Hough Transforms
      \end{enumerate}
  \end{enumerate}

  &

  \begin{enumerate}[leftmargin=*]
    \item Projections
    \item Template Matching
    \item Connected Components
  \end{enumerate}
  
  &

  \begin{enumerate}[leftmargin=*]
    \item \acrfull{KNN}
    \item Neural Networks
    \item Support Vector Machines
    \item Statistical Moments
  \end{enumerate}

  &

  \begin{enumerate}[leftmargin=*]
    \item Simple Heuristics
    \item Grammar
  \end{enumerate} \\
  \bottomrule
\end{tabularx}
\caption{OMR Stages}
\end{table}
