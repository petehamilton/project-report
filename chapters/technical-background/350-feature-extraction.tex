\section{Component Features}

\subsection{Basic Features}

Some basic features extracted from a component are:

\begin{enumerate}
  \item X Position
  \item Y Position
  \item Height
  \item Width
  \item 
\end{enumerate}

In some cases, x and y position on the staff are used in classification but for my purposed I have decided to perform classification assuming no prior knowledge. There might be exercises or free-drawing where a child doesn't want to have to follow the rules which apply to a normal score of music but rather draw specific components.

\subsection{Vertical and Horizontal Holes}

The average number of vertical and horizontal holes is used by \cite{fujinaga1996adaptive} and are topological properties in that they're generally scale invariant and as defined by \cite{burger2009principles} ``do not describe the shape of a region in continuous terms; instead, they capture its structural properties".

Average vertical and horizontal holes can be calculated by analysing the columns and rows respectively and counting the number of segments which do not have a solid pixel in (the runs of 0s). For example, for average horizontal holes we would count the number of white segments per row and total them up, averaging by the number of rows. Similarly with columns for vertical holes.

\subsection{Moments}

I use low level moments in this project to extract information like region area and centroid coordinates, however I wasn't able to get to grips with some of the higher order moments used by \cite{fujinaga1996adaptive,arabelo} to represent rotation, skew and other properties which may go some way to explaining my failed classifier in \cref{fig:knn-stats} (\cref{sec:knn-stats}) as my implementation may have simply been wrong. Since I achieved good classification results I was content to admit my shortcomings and stick to using a downsampled after that.

\subsubsection{Ordinary Definition}

As in \cite{burger2009principles} a moment of the order $p, q$ for an image or region $I(x,y)$ can be defined by \cref{eqn:general-moment}.

\begin{lequation}\label{eqn:general-moment}
  m_{pq} = \sum_{x, y \in \mathbb{R}} I(x,y) \cdot x^p y^q
\end{lequation}

When dealing with binary regions (which may or may not be the case since resampling via linear interpolation produces non-binary values), you only consider the pixels of value 1 and so can simplify \cref{eqn:general-moment} to \cref{eqn:general-moment-simple}.

\begin{lequation}\label{eqn:general-moment-simple}
  m_{pq} = \sum_{x, y \in \mathbb{R}} x^p y^q
\end{lequation}

\subsubsection{Zeroth Order Moments}

Zeroth order moments can be used to calculate the total sum of grey area of a region. The zero moment can be calculated using \cref{eqn:0-moment} and we can see the result is intuitive - the area 

\begin{lequation}\label{eqn:0-moment}
  \text{area} = m_{00} = \sum_{x, y \in \mathbb{R}} x^0 y^0 = \sum_{x, y \in \mathbb{R}} 1
\end{lequation}

\subsubsection{First Order Moments}

The first order moments $m{01}$ and $m_{10}$ are use to obtain the centre of mass of a component at $(\bar{x}, \bar{y})$.

\begin{lequation}\label{eqn:1-moment-mass}
\bar{x} = \frac{m_{10}}{m_{00}}, \bar{y} = \frac{m_{01}}{m_{00}}
\end{lequation}

\subsection{Run Length Encoding}
\label{sec:tb-rle}

\acrfull{RLE} is something which is regularly mentioned in regard to OMR, it involves taking a pixel based image and converting what would be a huge amount of information into a more compact format by establishing ``runs'' of identical pixels which are in a contiguous block.

For a two dimensional binary image a run of pixels can be represented by it's row, column, value and run length \parencite[p. 27-28]{burger2009principles} as seen in \cref{table:rle-2d}

\begin{table}[H]

    \begin{tabularx}{\textwidth}{ X | X }
        \toprule
        Example & RLE $(\textbf{row}, \textbf{column}, \textbf{value}, \textbf{length})$\\
        \midrule

        $$\begin{bmatrix}
        1 & 2 & 2 & 3  \\
        3 & 3 & 3 & 1  \\
        1 & 1 & 5 & 5  \\
        5 & 5 & 2 & 2
        \end{bmatrix}$$

        &
        {[} (0, 0, 1, 1), (0, 1, 2, 2), (0, 3, 3, 1), \newline
        (1, 0, 3, 3), (1, 3, 1, 1), (2, 0, 1, 2), \newline
        (2, 2, 5, 2), (3, 0, 5, 2), (3, 2, 5, 2) {]} \\
        \ & \ \\
    \bottomrule
    \end{tabularx}

    \caption{2D Greyscale Image}
    \label{table:rle-2d}
\end{table}


\begin{table}[H]
    \begin{tabularx}{\textwidth}{ X | X }
        \toprule
        Example & RLE $(\textbf{value}, \textbf{length})$\\
        \midrule

        {[} 1, 2, 2, 3, 3, 3, 3, 1, 1, 1, 5, 5, 5, 5, 2, 2 {]}
        &
        {[} $(1, 1)$,
        $(2, 2)$,
        $(3, 4)$,
        $(1, 3)$,
        $(5, 4)$,
        $(2, 2)$ {]} \\
    \bottomrule
    \end{tabularx}

    \caption{1D Flattened Greyscale Image}
    \label{table:rle-1d}
\end{table}

\begin{table}[H]
    \begin{tabularx}{\textwidth}{ X | X }
        \toprule
        Example & RLE $\langle \textbf{value}, \textbf{length} \rangle$\\
        \midrule

        {[} 0, 0, 0, 1, 1, 1, 0, 0, 1, 1, 1, 1 {]}
        &
        {[} 0, 3, 3, 2, 4 {]} (first bit sets ordering)\newline
        {[} 3, 3, 2, 4 {]} (assuming initial bit is 0) \\
    \bottomrule
    \end{tabularx}

    \caption{1D Flattened Binary Image}
    \label{table:rle-1d-binary}
\end{table}

If you then reshape this 2D image into a one dimensional array (retrieving and reshaping it to it's 2D representation later), you can remove 50\% of the compressed data (row and column) as seen in \cref{table:rle-1d}

If we are using binary data we can simplify this further, simply tracking the initial bit, then recording runs of alternating values as seen in \cref{table-1d-binary}, we can simplify this further with the additional assumption \parencite{fujinaga1996adaptive} that the sequence will start with a 0, removing the need for the initial bit. If the sequence begins with a 1 we just start with an entry of length 0. Since a lot of musical entities don't touch the top left corner, this is the implementation I have used.
