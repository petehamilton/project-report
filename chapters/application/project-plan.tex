\subsection{Detailed Plan \& Progress Tracker}

\begin{longtable}{|p{25mm}|p{61mm}|p{15mm}|p{50mm}|}
\hline
Date & Component & Status & Notes \\ \hline \hline

2014-01-21 & Implement tablet input prototype & DONE & \\ \hline
2014-01-21 & Implement simple image collection service & DONE & \\ \hline
2014-01-21 & Combine image collection service and input prototype & DONE & \\ \hline
2014-01-27 & Evaluate potential image analysis methods using small prototypes & DONE & Diff, Feature Matching (SIFT), Template Matching, Projection \\ \hline
\textbf{2014-01-30} & \textbf{Interim Project Report} & DONE & \\ \hline
2014-02-07 & Further investigation into graph similarity matching & DOING & \\ \hline
2014-02-14 & Investigation into image scoring using NCC & DOING & \\ \hline
2014-02-18 & Establish limits of analysis & - & Crochet vs Complex notation\\ \hline
\textbf{2014-02-20} & \textbf{Project Review} & - & \\ \hline
2014-03-05 & Input application completion & - & \\ \hline
2014-03-07 & Begin data gathering from test subjects & - & \\ \hline
2014-04-07 & Complete feedback and analysis component & - & \\ \hline
2014-04-07 & Begin user testing & - & One month of user testing \& tweaking \\ \hline
2014-04-14 & Data Analysis & - & Start in depth analysis after 2 weeks of testing \\ \hline
2014-05-01 & Final Project Report Write Up & - & Allow 3+ weeks \\ \hline
\textbf{2014-06-17} & \textbf{REPORT DUE} & - & \\ \hline
\textbf{2014-06-23} & \textbf{Preliminary Archive} & - & \\ \hline
\textbf{2014-06-30} & \textbf{Final Archive} & - & \\ \hline

\end{longtable}

Note: I have deliberately left myself free during weeks 2 and 3 of March in order to prepare for exams and I anticipate little progress during this time.


\subsection{Current Progress}
Currently I have built a prototype for the notation input on a tablet (Apple iPad) and in a web browser for easier testing.

I have established several techniques for image analysis and scoring which I am now in the process of evaluating further.

\subsection{Extensions}

If everything goes well, I would ideally like to be able to give users complex notation challenges such as combinations of quavers and crotchets, beamed notes, dotted notes, occasional accidentals all in one go.

I would also ideally like to make the gamification of encouragement and feedback adaptive to the users' progress.

\subsection{Fall-Back Options}

My most basic project aim is to be able to take input for crochets, quavers, minims, semibreves, accidentals and beamed notes. However I'm hopeful I will be able to go beyond this!
