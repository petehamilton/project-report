\section{Classification}

Classifiers are created by taking a sets of previously labelled samples and building a model which attempts to provide the most accurate relation between the samples and their labels.

This allows new samples to be classified using the model to apply the most likely (and hopefully correct) label.

\subsection{K Nearest Neighbour}

The \acrfull{KNN} algorithm is an example of instance based learning and theoretically doesn't require supervision or training.

KNN classification works by extracting a set of features for each sample and using this `feature vector' to represent each sample in a multidimensional vector space.

Once this space has been generated, new samples may be classified by generating their feature vector and calculating the $k$ closest or `most similar' samples in the model by way of a distance calculation. By considering the $k$ closest samples, a consensus can be reached and the new sample is classified according to the label with the greatest majority in the $k$ nearest neighbours as seen in \ref{fig:knn-example}.

% \begin{figure}[h!]
%   \centering
%   \includegraphics[draft, width=\linewidth]{gfx/technical-background/knn-example.png}
%   \caption{KNN example using musical symbols}
%   \label{fig:knn-example}
% \end{figure}
\todo[inline]{KNN EXAMPLE}
The two most critical aspects of the KNN classifier are the distance method used and the number of neighbours considered. In order to maximise the accuracy of the classifier I perform several experiments to this effect in section \ref{sec:imp-classification}
