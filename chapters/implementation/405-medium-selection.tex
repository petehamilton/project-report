\section{Medium Selection}
\label{sec:medium-selection}

The first step in this project was to establish the best method for a student to interact with the application. As seen in \ref{sec:prior-research} a lot of different mediums for OMR have been tried over the years, however which one would be most suitable for student use wouldn't necessarily correlate with which was best for quality of scanning, speed of analysis etc.

\subsection{Flat Bed Scanner}

The most simple of all the input methods, this would involve a student writing on a sheet of manuscript paper, then using a flat bed scanner (the type commonly found as a standalone device and in multi-function printers) to input the sheet into the computer. This method allows for a scan with high and consistent light levels (minimising noise and light-dependent artefacts), minimal distortion due to paper curvature (as the scanner usually flattens the sheet) and a high resolution end image, typically 300-600dpi.

The application would process the sheet using traditional OMR techniques and then provide feedback to the student. This technique would therefore require ownership of both a scanner and a computer on which to install and run the application. Alternatively, the student could upload the scanned image to a web based service for analysis, removing the need to install software.

\subsection{Photograph/Camera}

Total free for all, colours and light levels vary, extraction from background, resolution variable, noise variable

Project 8 Final Report

\subsection{Gestures}

Use gestures combined with classification (see book and some papers).
Good for classifying individual entities but more like the old palm tablets, usually one note at a time


\subsection{Tablet Input}

Finger difficult, stylus better, not always available

\subsection{Comparison Matrix}
