\section{Existing Approaches to OMR}

OMR strategies only varied slightly between the many projects I looked at and in general, the sequence was as follows:

\begin{enumerate}
\item Low Level Pre-Processing
\item Segmentation
\item Classification
\item Post Processing
\end{enumerate}

\subsection{Pre Processing}
Most OMR applications and research is focussed on processing scanned in printed music sheets. Usually this consists of taking scanning a sheet of A4 into a Grey (256 colors) colorspace at a reasonably high resolution of about 300 - 400dpi\footnote{dpi - Dots per inch}.

The image is generally then processed to remove any noise, skewing, warping, rotation or other graphical defects and then binarized, resulting in a black and white replica of the original score.

Finally, staff lines are generally removed \todo[inline]{(Bainbridge and Bell, 2001)} although this isn't always the case.

\subsection{Segmentation}
\subsection{Classification}
\subsection{Post Processing}
