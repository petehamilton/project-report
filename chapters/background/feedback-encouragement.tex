This project is primarily aimed at the very early stages of music theory and manuscript writing for young children. However, encouraging children to practice skills which require concentration can be challenging and so I hope to overcome this and persuade them to use the application by incorporating elements of gamification into it. This is something which has long been used in the field of HCI\footnote{Human Computer Interaction} to improve the user experience when using applications and has even been referred to as `funology' \parencite{blythe2004funology}

\subsection*{Badges \& Achievements}
It has been shown that you can encourage user behaviour with achievements, for example the site StackOverflow\footnote{http://www.stackoverflow.com - A Q\&A site for developers} uses badges\footnote{http://stackoverflow.com/help/badges} in order to encourage users who contribute to the site by answering a question, something which has been shown to influence the user's behaviour \parencite{MSR13p65}.

In a similar way, gamification is becoming used more and more in education. It is mentioned by \citeauthor{kapp2012gamification}\cite{kapp2012gamification}, that there are two main forms of motivation, intrinsically motivated behaviour (\enquote{when the rewards come from carrying out an activity rather than from the result of the activity}) and extrinsically motivated behaviour (\enquote{behaviour undertaken in order to obtain some reward or avoid punishment}).

Clearly out of the two, the former is preferable, as a self-motivated learner is always desirable in an educational context, however, for small children, mild application of positive extrinsic motivation can be very effective, especially when it rewards the effort involved, not just the outcome \parencite{motivate-kids-lifehacker}\cite{motivate-kids-lifehacker}
