\subsection{Mistake Detection and Scoring}

\emph{Objective}: \\ 
Enable the detection of common errors in the correct application of symbols, pitch, time signatures etc

\emph{Evaluation Technique}: \\
Compare automated scoring to that of a teacher, should identify the same mistakes

\subsection{Image Analysis}

\emph{Objective}: \\
Build a service which is capable of analysing a notation drawing and establishing one or more scores for multiple aspects of a notation attempt based on a reference example and/or previous attempts.

\emph{Evaluation Technique}: \\
Automated comparison against a test data set which has been evaluated by a professional music teacher.
Analysis of scoring and feedback by professional music teacher given written and verbally.

\subsection{Tablet Interface}

\emph{Objective}: \\
A tablet oriented (or tablet compatible) user interface which can provide a child with simulated challenges around writing musical notation and easily allow them to input their attempts and display feedback.

\emph{Evaluation Technique}: \\
Verbal and written feedback from child on their experience using the tablet for input.

Verbal and written feedback from teachers on whether the tablet accurately reflects writing normal manuscript notation

\subsection{Engaging Experience For Child}

\emph{Objective}: \\
Combine the two objectives above to produce a streamlined experience which a child will happily engage with on a repeat basis

\emph{Evaluation Technique}: \\
Short survey of each child after one week of use on criteria like ``Usability'', ``Fun-Factor'' and ``Likelihood of Repeat Use'' and analysis of usage logs for the application for each child.

\subsection{Learning Improvement}

\emph{Objective}: \\
Produce an application which can improve on and continue a child's learning outside of lessons

\emph{Evaluation Technique}: \\
Feedback from the user's (child's) teacher on the impact the application has on a child's performance over a 4 week period.