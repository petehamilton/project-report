\section{Scoring}

\subsection{Key Signatures}
\todo[inline,color=red]{Scoring Techniques - Key Signatures}

\subsection{Beats and Timing}
\todo[inline,color=red]{Scoring Techniques - Beats and Timing}

\subsection{Stems}
\subsubsection{Straightness}
\todo[inline,color=red]{Scoring Techniques - Stem Straightness}
\begin{figure}[h!]
  \centering
  \tiny
  \captionsetup[subfigure]{labelformat=empty}

  \subcaptionbox{7.3\label{fig:stem-straightness-4912}}[0.04\linewidth]{
    \includegraphics[height=1.5cm,keepaspectratio]{gfx/implementation/stem-straightness/4912.png}
  }
  \subcaptionbox{3.3\label{fig:stem-straightness-3889}}[0.04\linewidth]{
    \includegraphics[height=1.5cm,keepaspectratio]{gfx/implementation/stem-straightness/3889.png}
  }
  \subcaptionbox{3.1\label{fig:stem-straightness-4064}}[0.04\linewidth]{
    \includegraphics[height=1.5cm,keepaspectratio]{gfx/implementation/stem-straightness/4064.png}
  }
  \subcaptionbox{2.7\label{fig:stem-straightness-5371}}[0.04\linewidth]{
    \includegraphics[height=1.5cm,keepaspectratio]{gfx/implementation/stem-straightness/5371.png}
  }
  \subcaptionbox{2.5\label{fig:stem-straightness-3796}}[0.04\linewidth]{
    \includegraphics[height=1.5cm,keepaspectratio]{gfx/implementation/stem-straightness/3796.png}
  }
  \subcaptionbox{2.4\label{fig:stem-straightness-3845}}[0.04\linewidth]{
    \includegraphics[height=1.5cm,keepaspectratio]{gfx/implementation/stem-straightness/3845.png}
  }
  \subcaptionbox{2.2\label{fig:stem-straightness-4055}}[0.04\linewidth]{
    \includegraphics[height=1.5cm,keepaspectratio]{gfx/implementation/stem-straightness/4055.png}
  }
  \subcaptionbox{1.8\label{fig:stem-straightness-4072}}[0.04\linewidth]{
    \includegraphics[height=1.5cm,keepaspectratio]{gfx/implementation/stem-straightness/4072.png}
  }
  \subcaptionbox{1.6\label{fig:stem-straightness-4758}}[0.04\linewidth]{
    \includegraphics[height=1.5cm,keepaspectratio]{gfx/implementation/stem-straightness/4758.png}
  }
  \subcaptionbox{1.5\label{fig:stem-straightness-4653}}[0.04\linewidth]{
    \includegraphics[height=1.5cm,keepaspectratio]{gfx/implementation/stem-straightness/4653.png}
  }
  \subcaptionbox{1.4\label{fig:stem-straightness-4183}}[0.04\linewidth]{
    \includegraphics[height=1.5cm,keepaspectratio]{gfx/implementation/stem-straightness/4183.png}
  }
  \subcaptionbox{1.2\label{fig:stem-straightness-4250}}[0.04\linewidth]{
    \includegraphics[height=1.5cm,keepaspectratio]{gfx/implementation/stem-straightness/4250.png}
  }
  \subcaptionbox{1.1\label{fig:stem-straightness-4003}}[0.04\linewidth]{
    \includegraphics[height=1.5cm,keepaspectratio]{gfx/implementation/stem-straightness/4003.png}
  }
  \subcaptionbox{0.9\label{fig:stem-straightness-5350}}[0.04\linewidth]{
    \includegraphics[height=1.5cm,keepaspectratio]{gfx/implementation/stem-straightness/5350.png}
  }
  \subcaptionbox{0.8\label{fig:stem-straightness-4103}}[0.04\linewidth]{
    \includegraphics[height=1.5cm,keepaspectratio]{gfx/implementation/stem-straightness/4103.png}
  }
  \subcaptionbox{0.7\label{fig:stem-straightness-5104}}[0.04\linewidth]{
    \includegraphics[height=1.5cm,keepaspectratio]{gfx/implementation/stem-straightness/5104.png}
  }
  \subcaptionbox{0.6\label{fig:stem-straightness-4038}}[0.04\linewidth]{
    \includegraphics[height=1.5cm,keepaspectratio]{gfx/implementation/stem-straightness/4038.png}
  }
  \subcaptionbox{0.5\label{fig:stem-straightness-4565}}[0.04\linewidth]{
    \includegraphics[height=1.5cm,keepaspectratio]{gfx/implementation/stem-straightness/4565.png}
  }
  \subcaptionbox{0.4\label{fig:stem-straightness-4011}}[0.04\linewidth]{
    \includegraphics[height=1.5cm,keepaspectratio]{gfx/implementation/stem-straightness/4011.png}
  }
  \subcaptionbox{0.3\label{fig:stem-straightness-3327}}[0.04\linewidth]{
    \includegraphics[height=1.5cm,keepaspectratio]{gfx/implementation/stem-straightness/3327.png}
  }
  \subcaptionbox{0.2\label{fig:stem-straightness-4752}}[0.04\linewidth]{
    \includegraphics[height=1.5cm,keepaspectratio]{gfx/implementation/stem-straightness/4752.png}
  }
  \subcaptionbox{0.1\label{fig:stem-straightness-3876}}[0.04\linewidth]{
    \includegraphics[height=1.5cm,keepaspectratio]{gfx/implementation/stem-straightness/3876.png}
  }
  \caption{Ranking of stems according to their `straightness' score}
  \label{fig:stem-straightness-ranking}
\end{figure}


\subsubsection{Angle}
\todo[inline,color=red]{Scoring Techniques - Stem Angle}
 \begin{figure}[h!]
        \centering
        \tiny
        \addtolength{\subfigcapskip}{0.2cm}

        \subfigure[-23.9]{
          \includegraphics[height=2cm,keepaspectratio]{gfx/implementation/stem-angle/4883.png}
          \label{fig:stem-angle-4883}
        }
        \quad
        \subfigure[-20.0]{
          \includegraphics[height=2cm,keepaspectratio]{gfx/implementation/stem-angle/4718.png}
          \label{fig:stem-angle-4718}
        }
        \quad
        \subfigure[18.7]{
          \includegraphics[height=2cm,keepaspectratio]{gfx/implementation/stem-angle/4705.png}
          \label{fig:stem-angle-4705}
        }
        \quad
        \subfigure[-14.7]{
          \includegraphics[height=2cm,keepaspectratio]{gfx/implementation/stem-angle/4886.png}
          \label{fig:stem-angle-4886}
        }
        \quad
        \subfigure[-9.8]{
          \includegraphics[height=2cm,keepaspectratio]{gfx/implementation/stem-angle/4892.png}
          \label{fig:stem-angle-4892}
        }
        \quad
        \subfigure[-8.4]{
          \includegraphics[height=2cm,keepaspectratio]{gfx/implementation/stem-angle/4645.png}
          \label{fig:stem-angle-4645}
        }
        \quad
        \subfigure[7.3]{
          \includegraphics[height=2cm,keepaspectratio]{gfx/implementation/stem-angle/3889.png}
          \label{fig:stem-angle-3889}
        }
        \quad
        \subfigure[6.2]{
          \includegraphics[height=2cm,keepaspectratio]{gfx/implementation/stem-angle/5380.png}
          \label{fig:stem-angle-5380}
        }
        \quad
        \subfigure[5.1]{
          \includegraphics[height=2cm,keepaspectratio]{gfx/implementation/stem-angle/4194.png}
          \label{fig:stem-angle-4194}
        }
        \quad
        \subfigure[-4.1]{
          \includegraphics[height=2cm,keepaspectratio]{gfx/implementation/stem-angle/3970.png}
          \label{fig:stem-angle-3970}
        }
        \quad
        \subfigure[3.0]{
          \includegraphics[height=2cm,keepaspectratio]{gfx/implementation/stem-angle/4783.png}
          \label{fig:stem-angle-4783}
        }
        \quad
        \subfigure[2.0]{
          \includegraphics[height=2cm,keepaspectratio]{gfx/implementation/stem-angle/3920.png}
          \label{fig:stem-angle-3920}
        }
        \quad
        \subfigure[-1.0]{
          \includegraphics[height=2cm,keepaspectratio]{gfx/implementation/stem-angle/5368.png}
          \label{fig:stem-angle-5368}
        }
        \quad

        \caption{Ranking of stems according to their off-vertical angle}
        \label{fig:stem-angle-ranking}
      \end{figure}



\subsubsection{Direction}

To establish the stem direction which I will refer to as $S_d$, the relative upper left coordinates of the head and the stem are compared.

If the stem is located at $(x_{\text{stem}}, y_{\text{stem}})$ and the note head at $(x_{\text{head}}, y_{\text{head}})$, give the coordinate axes have an origin starting from the top right of a given image, we can establish the following classifications:

$$
S_{d} (y_{\text{stem}}, y_{\text{head}}) =
\left\{
	\begin{array}{ll}
		\text{up}   & \mbox{if } y_{\text{stem}} < y_{\text{head}} \\
		\text{down} & \mbox{if } y_{\text{stem}} > y_{\text{head}}
	\end{array}
\right.
$$

The case where a stem has the same $y$ coordinate as it's head isn't possible due to the way stems are extracted from a note complex as for this to happen a stem would need to be extracted \emph{out} of a note head. Since note heads are removed in order to identify stems as in section \todo[inline, color=blue]{REFERENCE: Extracting Stems}, this is impossible.

\subsubsection{Side}
To establish the stem side which I will refer to as $S_s$, we do a similar operation to in determining the direction, however, since it is perfectly feasible that the stem and head have the same x coordinate, we can't just use the x coordinate directly.

Instead, we compare the x coordinate of the stem $x_{\text{stem}}$ to the true centre \todo{Better definition later please} of the note head $cx_{\text{head}}$

Again, given that the coordinate axes start from the top left, we can now establish the stem x offset

$$
S_\text{xoff} = x_{\text{stem}} - cx_{\text{head}}
$$

Using this, we can establish the side classifcation as
$$
S_{s} (S_\text{xoff}) =
\left\{
	\begin{array}{ll}
		\text{left}   & \mbox{if } S_\text{xoff} < 0 \\
		\text{right}  & \mbox{if } S_\text{xoff} \gte 0
	\end{array}
\right.
$$

\todo[inline,color=red]{Scoring Techniques - Stem Side}

\subsubsection{Length}
\todo[inline,color=red]{Scoring Techniques - Stem Length}

\subsection{Quaver Tails}
\todo[inline,color=red]{Scoring Techniques - Quaver tail side}

\subsection{Note Heads}
\todo[inline,color=red]{Scoring Techniques - Filled Percentage? (ambiguity)}
\todo[inline,color=red]{Scoring Techniques - Stem Direction}
\todo[inline,color=red]{Scoring Techniques - Stem Side}
\todo[inline,color=red]{Scoring Techniques - Stem Straightness}

\subsection{Aggregation}
Given the large number of variables outlined previously, to produce a simple numeric score we must somehow aggregate the more detailed feedback. For example, we could assign weights to `musical correctness', `stems', `note heads' etc, then simply produce a score in a reasonable range for feedback.