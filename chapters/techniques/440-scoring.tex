\section{Scoring and Feedback}
\label{sec:scoring}

\subsection{Pitched Notes and Accidentals}
\label{sec:pitched-notes-accidentals}

\subsubsection{Position}\label{sec:scoring-position}

For pitched notes and accidentals, we compare their vertical centres as calculated in the identification (\cref{sec:identification}) stages to positions of the surrounding stave lines and spaces.

We designate the attempted stave position as being the one which is closest to this center (the same position which was used for calculating pitch) and then measure the distance from it. If the distance is greater than a given threshold (obtained experimentally) we then designate this a potentially ambiguous entity.

\subsection{Key Signatures}

A key signature consists of several sharps or flats which are required to be in a certain order and position on the stave.

Before we perform any analysis, we need a reference copy of what we expect the accidentals in the key signature to look like. We therefore count the number of drawn accidentals and then produce the set of accidentals we would expect to see for comparison. Once this has been done there are several analysis methods we can use to further assess the key signature which are summarised in \cref{table:key-signature-errors}.

\begin{table}[H]
    \renewcommand{\arraystretch}{1.6}
    \begin{tabularx}{\textwidth}{ lX }
        \toprule
        Mistake & Feedback \\
        \midrule
        Mixed Accidentals & Incorrect because sharps can't be mixed \\
        Wrong Octave & Which sharps were affected and whether they're an octave too high or too low \\
        Incorrect Order & Which sharps are in the wrong place and where they should be \\
        Incorrect Accidentals & Which accidentals are wrong and what they should be \\
        \bottomrule
    \end{tabularx}

    \caption{Mistakes we need to look for in key signatures}
    \label{table:key-signature-errors}
\end{table}

\subsubsection{Mixed Accidentals}\label{sec:scoring-keysig-mixed}

Key signatures (of the type we're concerned with in the project) may contain only sharps or flats, not a combination of both. If a key signature is detected to contain both sharps and flats such as in \cref{fig:mixed-accidentals-bad}, the fact that this has been done is noted, but at this point no further analysis is done.

\begin{figure}[H]
  \missingfigure{Sharps and Flats Mixed}
  \caption{}
  \label{fig:mixed-accidentals-bad}
\end{figure}

\subsubsection{Wrong Octaves}\label{sec:scoring-keysig-octave}

Sometimes the student gets the right accidental, but just in the wrong octave, for example the case identified by a teacher in \cref{fig:right-accidental-wrong-octave}. This can be identified by checking to see if the actual note (C, D, E) the drawn accidental rather than a specific pitch (C4, D4, E4) matches that of the expected accidental.

\begin{figure}[H]
  \missingfigure{Right accidental wrong octave teacher example}
  \caption{}
  \label{fig:right-accidental-wrong-octave}
\end{figure}

\subsubsection{Out of Order}\label{sec:scoring-keysig-order}

Sometimes a key signature contains all the sharps required, but not in the right order as seen in \cref{fig:right-accidentals-wrong-ordering}. This case can be established by taking the number of accidentals drawn, establishing what would be the correct accidentals for that case and comparing the list of pitches to an ordered list of the drawn pitches. If they match, the student has mis-ordered some of the accidentals so we note which ones are out of order and where they should be.

\begin{figure}[H]
  \missingfigure{Right accidentals wrong order example}
  \caption{}
  \label{fig:right-accidentals-wrong-ordering}
\end{figure}

\subsubsection{Incorrect Accidental}\label{sec:scoring-keysig-incorrect}


\subsection{Beats and Timing}\label{sec:scoring-beats}

There are really only two things we need to check for when scoring beats and timing, outlined in \cref{table:bar-beat-errors}.

\begin{table}[H]
    \renewcommand{\arraystretch}{1.6}
    \begin{tabularx}{\textwidth}{ lX }
        \toprule
        Mistake & Feedback to relay \\
        \midrule
        Too few beats in a bar & Which bar, how many beats we count and how many we expect based on the time signature \\
        Too many beats in a bar & Which bar, how many beats we count and how many we expect based on the time signature \\
        \bottomrule
    \end{tabularx}

    \caption{Mistakes we need to look for in beats per bar}
    \label{table:bar-beat-errors}
\end{table}

Regardless of which mistake we're checking for the algorithm is essentially the same and can be checked at the same time as demonstrated in \cref{alg:bar-beat-check}

\begin{algorithm}[H]
  \caption{Searching for incorrect beats per bar}
  \label{alg:bar-beat-check}

  \begin{algorithmic}
    \Procedure{CheckBars}{$bars$}
        \For {each bar in stave.bars}
          \State expectedBeats = \Call{GetExpectedBeatsInBar}{bar}
          \State actualBeats = \Call{GetActualBeatsInBar}{bar}
          \If{actualBeats $>$ expectedBeats}
            \State{Log the mistake}
          \ElsIf{actualBeats $<$ expectedBeats}
            \State{Log the mistake}
          \EndIf
        \EndFor
    \EndProcedure
    \Statex
    \Procedure{GetActualBeatsInBar}{$bar$}
        \State totalBeats = 0
        \For {each note in bar.notes}
          \State totalBeats += \Call{GetNoteLength}{note}
        \EndFor

        \Return totalBeats
    \EndProcedure
  \end{algorithmic}
\end{algorithm}

\subsection{Stems}

Stems are a very common component of manuscript and as such will be drawn regularly. It's important to have detailed checking to weed out any bad habits early and so we check for the all the potential mistakes in \cref{table-stem-mistakes}.

\begin{table}[H]
    \renewcommand{\arraystretch}{1.6}
    \begin{tabularx}{\textwidth}{ lX }
        \toprule
        Mistake & Feedback \\
        \midrule
        Straightness & Whether it's just a bit uneven or really wonky \\
        Angle & Whether the stem was leaning just a little bit or severely \\
        Direction & Which direction the stem should have been pointing \\
        Side & Which side the stem should have been on \\
        Length & Whether it was too long or too short\\
        \bottomrule
    \end{tabularx}

    \caption{Mistakes we need to look for in note stems}
    \label{table:stem-mistakes}
\end{table}

\subsubsection{Straightness}\label{sec:scoring-stem-straightness}

Given a drawn note stem, we wish to be able to determine a measure of `straightness' which we can threshold to discern a badly drawn stem shown in \cref{fig:wonky-stems} from a straight one  as shown in \cref{fig:straight-stems}.

\begin{figure}[h!]
    \centering
    \begin{subfigure}[b]{.4\linewidth}
        \centering
        \includegraphics[height=4cm]{gfx/implementation/stem-straightness/4912.png}
        \quad
        \includegraphics[height=4cm]{gfx/implementation/stem-straightness/5371.png}
        \caption{Examples of uneven stems}
        \label{fig:wonky-stems}
    \end{subfigure}
    \begin{subfigure}[b]{.4\linewidth}
        \centering
        \includegraphics[height=4cm]{gfx/implementation/stem-straightness/3876.png}
        \quad
        \includegraphics[height=4cm]{gfx/implementation/stem-straightness/5104.png}
        \caption{Examples of straight stems}
        \label{fig:straight-stems}
    \end{subfigure}

    \caption{Examples of straight and uneven stems}
    \label{fig:stem-straightness-examples}
\end{figure}

Stem straightness is different to the stem angle because intuitively even if the stem is angled, it can still be straight. Therefore, we need to establish a technique which gives a measure irrespective of the stem angle.

To do this, we take the original stem (\cref{fig:straight-skeleton-original}) and generate it's skeletal representation using techniques outlined in \cref{sec:skeletonization} which approximates a line following the center of the stem (\cref{fig:straight-skeleton-skeleton}). If we treat this skeleton as a plot of points, we can draw a line of best fit through them to approximate what a perfectly straight version of the stem (\cref{fig:straight-skeleton-bestfit}).

\begin{figure}[h!]
    \centering

    \begin{subfigure}[b]{.3\linewidth}
        \centering
        \frame{\includegraphics[height=3cm]{gfx/techniques/skeletonization/4705.png}}
        \frame{\includegraphics[height=3cm]{gfx/techniques/skeletonization/4912.png}}
        \caption{Original}
        \label{fig:straight-skeleton-original}
    \end{subfigure}
    \begin{subfigure}[b]{.3\linewidth}
        \centering
        \frame{\includegraphics[height=3cm]{gfx/techniques/skeletonization/4705_skeleton.png}}
        \frame{\includegraphics[height=3cm]{gfx/techniques/skeletonization/4912_skeleton.png}}
        \caption{Skeleton}
        \label{fig:straight-skeleton-skeleton}
    \end{subfigure}
    \begin{subfigure}[b]{.3\linewidth}
        \centering
        \frame{\includegraphics[height=3cm]{gfx/techniques/skeletonization/4705_bestfit.png}}
        \frame{\includegraphics[height=3cm]{gfx/techniques/skeletonization/4912_bestfit.png}}
        \caption{Best Fit}
        \label{fig:straight-skeleton-bestfit}
    \end{subfigure}

    \caption{Examples of stem skeletons}
    \label{fig:stem-skeletons}
\end{figure}

We now have a skeleton line \cref{eqn:l-skeleton} and a best fit line \cref{eqn:l-ref} we can calculate the difference at each point \cref{eqn:l-residuals}.

\begin{lequation} \label{eqn:l-skeleton}
    L_{\text{skeleton}}(x) = (a_0, a_1, a_2, \ldots, a_n)
\end{lequation}
\begin{lequation} \label{eqn:l-ref}
    L_{\text{ref}}(x) = (b_0, b_1, b_2, \ldots, b_n)
\end{lequation}
\begin{lequation} \label{eqn:l-residuals}
    R(x) = L_{\text{skeleton}}(x) - L_{\text{ref}}(x) = (r_0, r_1, r_2, \ldots, r_n)
\end{lequation}

\begin{lequation} \label{eqn:sd}
\text{Straightness} = \sigma = \sqrt{\frac{\sum\limits_{i=1}^{n}\left(r_{i} - \bar{r}\right)^{2}}{n-1}}
\end{lequation}

After experimenting with using the standard deviation \cref{eqn:sd} of the residuals as the straightness measure the results turned out to be positive upon visual inspection. As you can see in \cref{fig:stem-straightness-ranking} when ranked according to their `straightness' measure, the stems do indeed appear to be ordered according to what one would visually define as being `straight'.

\begin{figure}[h!]
  \centering
  \tiny
  \captionsetup[subfigure]{labelformat=empty}

  \subcaptionbox{7.3\label{fig:stem-straightness-4912}}[0.04\linewidth]{
    \includegraphics[height=1.5cm,keepaspectratio]{gfx/implementation/stem-straightness/4912.png}
  }
  \subcaptionbox{3.3\label{fig:stem-straightness-3889}}[0.04\linewidth]{
    \includegraphics[height=1.5cm,keepaspectratio]{gfx/implementation/stem-straightness/3889.png}
  }
  \subcaptionbox{3.1\label{fig:stem-straightness-4064}}[0.04\linewidth]{
    \includegraphics[height=1.5cm,keepaspectratio]{gfx/implementation/stem-straightness/4064.png}
  }
  \subcaptionbox{2.7\label{fig:stem-straightness-5371}}[0.04\linewidth]{
    \includegraphics[height=1.5cm,keepaspectratio]{gfx/implementation/stem-straightness/5371.png}
  }
  \subcaptionbox{2.5\label{fig:stem-straightness-3796}}[0.04\linewidth]{
    \includegraphics[height=1.5cm,keepaspectratio]{gfx/implementation/stem-straightness/3796.png}
  }
  \subcaptionbox{2.4\label{fig:stem-straightness-3845}}[0.04\linewidth]{
    \includegraphics[height=1.5cm,keepaspectratio]{gfx/implementation/stem-straightness/3845.png}
  }
  \subcaptionbox{2.2\label{fig:stem-straightness-4055}}[0.04\linewidth]{
    \includegraphics[height=1.5cm,keepaspectratio]{gfx/implementation/stem-straightness/4055.png}
  }
  \subcaptionbox{1.8\label{fig:stem-straightness-4072}}[0.04\linewidth]{
    \includegraphics[height=1.5cm,keepaspectratio]{gfx/implementation/stem-straightness/4072.png}
  }
  \subcaptionbox{1.6\label{fig:stem-straightness-4758}}[0.04\linewidth]{
    \includegraphics[height=1.5cm,keepaspectratio]{gfx/implementation/stem-straightness/4758.png}
  }
  \subcaptionbox{1.5\label{fig:stem-straightness-4653}}[0.04\linewidth]{
    \includegraphics[height=1.5cm,keepaspectratio]{gfx/implementation/stem-straightness/4653.png}
  }
  \subcaptionbox{1.4\label{fig:stem-straightness-4183}}[0.04\linewidth]{
    \includegraphics[height=1.5cm,keepaspectratio]{gfx/implementation/stem-straightness/4183.png}
  }
  \subcaptionbox{1.2\label{fig:stem-straightness-4250}}[0.04\linewidth]{
    \includegraphics[height=1.5cm,keepaspectratio]{gfx/implementation/stem-straightness/4250.png}
  }
  \subcaptionbox{1.1\label{fig:stem-straightness-4003}}[0.04\linewidth]{
    \includegraphics[height=1.5cm,keepaspectratio]{gfx/implementation/stem-straightness/4003.png}
  }
  \subcaptionbox{0.9\label{fig:stem-straightness-5350}}[0.04\linewidth]{
    \includegraphics[height=1.5cm,keepaspectratio]{gfx/implementation/stem-straightness/5350.png}
  }
  \subcaptionbox{0.8\label{fig:stem-straightness-4103}}[0.04\linewidth]{
    \includegraphics[height=1.5cm,keepaspectratio]{gfx/implementation/stem-straightness/4103.png}
  }
  \subcaptionbox{0.7\label{fig:stem-straightness-5104}}[0.04\linewidth]{
    \includegraphics[height=1.5cm,keepaspectratio]{gfx/implementation/stem-straightness/5104.png}
  }
  \subcaptionbox{0.6\label{fig:stem-straightness-4038}}[0.04\linewidth]{
    \includegraphics[height=1.5cm,keepaspectratio]{gfx/implementation/stem-straightness/4038.png}
  }
  \subcaptionbox{0.5\label{fig:stem-straightness-4565}}[0.04\linewidth]{
    \includegraphics[height=1.5cm,keepaspectratio]{gfx/implementation/stem-straightness/4565.png}
  }
  \subcaptionbox{0.4\label{fig:stem-straightness-4011}}[0.04\linewidth]{
    \includegraphics[height=1.5cm,keepaspectratio]{gfx/implementation/stem-straightness/4011.png}
  }
  \subcaptionbox{0.3\label{fig:stem-straightness-3327}}[0.04\linewidth]{
    \includegraphics[height=1.5cm,keepaspectratio]{gfx/implementation/stem-straightness/3327.png}
  }
  \subcaptionbox{0.2\label{fig:stem-straightness-4752}}[0.04\linewidth]{
    \includegraphics[height=1.5cm,keepaspectratio]{gfx/implementation/stem-straightness/4752.png}
  }
  \subcaptionbox{0.1\label{fig:stem-straightness-3876}}[0.04\linewidth]{
    \includegraphics[height=1.5cm,keepaspectratio]{gfx/implementation/stem-straightness/3876.png}
  }
  \caption{Ranking of stems according to their `straightness' score}
  \label{fig:stem-straightness-ranking}
\end{figure}


\subsubsection{Angle}\label{sec:scoring-stem-angle}

In order to determine the angle of a stem, we first need establish the line of best fit which most accurately reflects the direction of the stem, we can then examine this line $y = mx + c$ and compute $\arctan(m)$ to get the angle relative to vertical.

\begin{figure}[h!]
    \centering

    \begin{subfigure}[b]{.45\linewidth}
        \centering
        \frame{\includegraphics[height=6cm]{gfx/techniques/skeletonization/4705.png}}
        \caption{Original}
    \end{subfigure}
    \begin{subfigure}[b]{.45\linewidth}
        \centering
        \frame{\includegraphics[height=6cm]{gfx/techniques/skeletonization/4705_bestfit_angle.png}}
        \caption{Best fit line with angle calculation}
    \end{subfigure}

    \caption{Examples of stem skeletons}
    \label{fig:stem-skeletons}
\end{figure}

 \begin{figure}[h!]
        \centering
        \tiny
        \addtolength{\subfigcapskip}{0.2cm}

        \subfigure[-23.9]{
          \includegraphics[height=2cm,keepaspectratio]{gfx/implementation/stem-angle/4883.png}
          \label{fig:stem-angle-4883}
        }
        \quad
        \subfigure[-20.0]{
          \includegraphics[height=2cm,keepaspectratio]{gfx/implementation/stem-angle/4718.png}
          \label{fig:stem-angle-4718}
        }
        \quad
        \subfigure[18.7]{
          \includegraphics[height=2cm,keepaspectratio]{gfx/implementation/stem-angle/4705.png}
          \label{fig:stem-angle-4705}
        }
        \quad
        \subfigure[-14.7]{
          \includegraphics[height=2cm,keepaspectratio]{gfx/implementation/stem-angle/4886.png}
          \label{fig:stem-angle-4886}
        }
        \quad
        \subfigure[-9.8]{
          \includegraphics[height=2cm,keepaspectratio]{gfx/implementation/stem-angle/4892.png}
          \label{fig:stem-angle-4892}
        }
        \quad
        \subfigure[-8.4]{
          \includegraphics[height=2cm,keepaspectratio]{gfx/implementation/stem-angle/4645.png}
          \label{fig:stem-angle-4645}
        }
        \quad
        \subfigure[7.3]{
          \includegraphics[height=2cm,keepaspectratio]{gfx/implementation/stem-angle/3889.png}
          \label{fig:stem-angle-3889}
        }
        \quad
        \subfigure[6.2]{
          \includegraphics[height=2cm,keepaspectratio]{gfx/implementation/stem-angle/5380.png}
          \label{fig:stem-angle-5380}
        }
        \quad
        \subfigure[5.1]{
          \includegraphics[height=2cm,keepaspectratio]{gfx/implementation/stem-angle/4194.png}
          \label{fig:stem-angle-4194}
        }
        \quad
        \subfigure[-4.1]{
          \includegraphics[height=2cm,keepaspectratio]{gfx/implementation/stem-angle/3970.png}
          \label{fig:stem-angle-3970}
        }
        \quad
        \subfigure[3.0]{
          \includegraphics[height=2cm,keepaspectratio]{gfx/implementation/stem-angle/4783.png}
          \label{fig:stem-angle-4783}
        }
        \quad
        \subfigure[2.0]{
          \includegraphics[height=2cm,keepaspectratio]{gfx/implementation/stem-angle/3920.png}
          \label{fig:stem-angle-3920}
        }
        \quad
        \subfigure[-1.0]{
          \includegraphics[height=2cm,keepaspectratio]{gfx/implementation/stem-angle/5368.png}
          \label{fig:stem-angle-5368}
        }
        \quad

        \caption{Ranking of stems according to their off-vertical angle}
        \label{fig:stem-angle-ranking}
      \end{figure}




\subsubsection{Direction}\label{sec:scoring-stem-direction}

To establish the stem direction which I will refer to as $S_d$, the vertical position (y coordinates) of the head and the stem are compared.

If the stem is located at $(x_{\text{stem}}, y_{\text{stem}})$ and the note head at $(x_{\text{head}}, y_{\text{head}})$, knowing that the coordinate axes have an origin starting from the top right of a given image, we can establish the following classifications:

$$
S_{d} (y_{\text{stem}}, y_{\text{head}}) =
\left\{
    \begin{array}{ll}
        \text{up}   & \mbox{if } y_{\text{stem}} < y_{\text{head}} \\
        \text{down} & \mbox{if } y_{\text{stem}} > y_{\text{head}}
    \end{array}
\right.
$$

The case where a stem has the same $y$ coordinate as it's head isn't possible due to the way stems are extracted from a note complex as for this to happen a stem would need to be extracted \emph{out} of a note head. Since note heads are removed in order to identify stems as in \todo[color=blue]{REFERENCE: Extracting Stems}, this is impossible.

\begin{figure}[h!]
    \centering
    \includegraphics[height=6cm]{gfx/techniques/scoring/note-stem-up/4247.png}
    \quad
    \includegraphics[height=6cm]{gfx/techniques/scoring/note-stem-up/6028.png}
    \quad
    \includegraphics[height=6cm]{gfx/techniques/scoring/note-stem-up/6042.png}
    \quad
    \includegraphics[height=6cm]{gfx/techniques/scoring/note-stem-up/6111.png}

    \caption{Identifying upward stems, the red line represents the stem y  coordinate and the blue represents the head's y coordinate}
    \label{fig:downward-stem-identification}
\end{figure}

\begin{figure}[h!]
    \centering
    \includegraphics[height=6cm]{gfx/techniques/scoring/note-stem-down/1747.png}
    \quad
    \includegraphics[height=6cm]{gfx/techniques/scoring/note-stem-down/2996.png}
    \quad
    \includegraphics[height=6cm]{gfx/techniques/scoring/note-stem-down/5327.png}
    \quad
    \includegraphics[height=6cm]{gfx/techniques/scoring/note-stem-down/5376.png}
    \quad
    \includegraphics[height=6cm]{gfx/techniques/scoring/note-stem-down/6090.png}

    \caption{Identifying downward stems, the red line represents the stem y  coordinate and the blue represents the head's y coordinate}
    \label{fig:downward-stem-identification}
\end{figure}

\subsubsection{Side}\label{sec:scoring-stem-side}
To establish the stem side which I will refer to as $S_s$, we do a similar operation to in determining the direction, however, since it is perfectly feasible that the stem and head have the same x coordinate, we can't just use the x coordinate directly.

Instead, we compare the x coordinate of the stem $x_{\text{stem}}$ to the true centre \todo[inline]{Better definition later please} of the note head $cx_{\text{head}}$

Again, given that the coordinate axes start from the top left, we can now establish the stem x offset

$$
S_\text{xoff} = x_{\text{stem}} - cx_{\text{head}}
$$

Using this, we can establish the side classification as
$$
S_{s} (S_\text{xoff}) =
\left\{
    \begin{array}{ll}
        \text{left}   & \mbox{if } S_\text{xoff} < 0 \\
        \text{right}  & \mbox{if } S_\text{xoff} \ge 0
    \end{array}
\right.
$$

\subsubsection{Length}\label{sec:scoring-stem-length}

Stem length may seem like a simple case of measuring the length of the extracted stem, but it's actually a little more complicated. In order to get the best results, we need to take into account the fact that the stem has been extracted from the note head and that this separation point could occur anywhere from near the bottom to the top as shown in \cref{fig:stem-intersection}.

\begin{figure}[h!]
    \centering
    \includegraphics[width=.8\textwidth]{gfx/techniques/stem-intersection-labelled.png}
    \caption{An example of the effect stem and note head intersection can have on length. Note that the right hand note would, if the raw stem was extracted, have a longer stem (potentially perceived as \emph{too} long) but in actual fact both stems reach the required height}
    \label{fig:stem-intersection}
\end{figure}

Similarly to the straightness and angular measures, we can establish the raw length $L_R$ by drawing a line through the perceived centre of the stem and measuring it's length using simple trigonometry. However, although it might seem counterintuitive to not use this as a measure of length, in reality a better measure of whether the stem is `too long' is to measure the distance between the stave line/space above stem's associated notehead and the y coordinate of the stem.

If we define the correct length $L_E$ for a stem as outlined in \todo[inline]{REFERENCE where I talk about correct stem length} then the length score is therefore just $L_E - L_A$ where smaller values mean the stem is closer to the expected length.

\subsection{Quaver Tail}

\subsubsection{Side}\label{sec:scoring-quaver-tail-side}

\todo[inline]{Quaver tail side needs a little more detail, this feels a bit `late in the evening' weak. Also, MORE MATHS}

To determine on which side a quaver tail lies, left (\cref{fig:quaver-tail-left}) or right (\cref{fig:quaver-tail-right}), intially I attempted to utilise vertical projections, much as in segmentation (\cref{sec:projections}).

\begin{figure}[h!]
    \centering

    \begin{subfigure}[b]{.45\linewidth}
        \centering
      \includegraphics[height=5cm]{gfx/techniques/quaver-left-6087.png}
      \caption{A left sided quaver tail}
      \label{fig:quaver-tail-left}
    \end{subfigure}
    \begin{subfigure}[b]{.45\linewidth}
        \centering
      \includegraphics[height=5cm]{gfx/techniques/quaver-right-3083.png}
      \caption{A right sided quaver tail}
      \label{fig:quaver-tail-right}
    \end{subfigure}

    \caption{Quaver tail sides}
\end{figure}

After taking a vertical projection of the quaver tail, the maximum peak in the projection is taken and by establishing on which side of the centre the peak lies, this told us which side the `flick' of the quaver is on.

However, occasionally a quaver tail would be encountered where the distinction wasn't very clear \todo[color=purple]{Example of non-distinct quaver tail position}.

Another method I investigated was to divide the image into two horizontally, then calculate the average black pixel column coordinate in each half as seen in \cref{fig:quaver-tail-average-position}. By working out which half has the highest average pixel position, we can then establish the side on which the tail lies.

% Generate the below with: python reporting_quavertail.py 3811
\begin{figure}[h!]
    \centering

    \begin{subfigure}[b]{.45\linewidth}
        \centering
      \frame{\includegraphics[height=7cm]{gfx/techniques/quaver-pixel-average-right.png}}
      \caption{Identifying a right sided tail}
      \label{fig:quaver-tail-average-position-right}
    \end{subfigure}
    \begin{subfigure}[b]{.45\linewidth}
        \centering
      \frame{\includegraphics[height=7cm]{gfx/techniques/quaver-pixel-average-left}}
      \caption{Identifying a left sided tail}
      \label{fig:quaver-tail-average-position-left}
    \end{subfigure}

      \caption{Calculating the quaver tail side with pixel averages}
      \label{fig:quaver-tail-average-position}
\end{figure}

\subsection{Note Heads}

\subsubsection{Size}\label{sec:scoring-note-head-size}
\todo[inline,color=red]{Scoring Techniques - Note Head - Size}

\subsubsection{Messy Crotchet Heads}\label{sec:scoring-note-head-messy}
\todo[inline,color=red]{Scoring Techniques - Note Head - Filled Percentage (e.g scruffy crotchets)}

\subsubsection{Broken Heads}\label{sec:scoring-note-head-broken}
\todo[inline,color=red]{Scoring Techniques - Note Head - Gaps/broken circles}
