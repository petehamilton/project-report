\section{Identification}

\todo[inline,color=red]{Identification Techniques - Write}

\subsection{Segmentation}

Why won't projection segmentation work for us? Want to be able to isolate stems and beams, projection segmentation only gives vertical sections


\subsection{Feature Extraction}

\subsubsection{Run Length Encoding}
To improve the speed of my application, I utilised \acrfull{RLE} (covered in \cref{sec:tb-rle}) to improve storage and retrieval times during feature extraction and classification by up to 97\%.

Testing with a sample set of 3000 stored components, improvements in storage and runtime are noted in table \cref{table:rle-improvement}

\begin{table}[h]

    \begin{tabularx}{\textwidth}{ X X X X }
    \toprule
    Metric                  & Without RLE   & With RLE   & Improvement \\
    \midrule
    Storage Required        & 759 MB        & 22 MB      & 91.7\%      \\
    Retrieval Time          & 1639 ms       & 42 ms      & 97.4\% \\
    \bottomrule
    \end{tabularx}

    \label{table:rle-improvement}
    \caption{Improvements after implementing RLE}
\end{table}

\subsection{Classification}
\label{sec:implementation-classification}

Classifiers are created by taking a sets of previously labelled samples and building a model which attempts to provide the most accurate relation between the samples and their labels.

This allows new samples to be classified using the model to apply the most likely (and hopefully correct) label.

\subsubsection{K Nearest Neighbour}

\subsection{Pitch Identification}
\label{sec:pitch-identification}

\subsubsection{Finding Component Centres}

For most components, the obvious choice would be to simply divide it in half vertically and horizontally and use that as an estimate for the centre and indeed that's often what is done in traditional OMR. \todo{Reference for dividing in half to get centre}.
\todo{I know someone uses 1/3 for flats}.

However, for several components I believe that a higher degree of accuracy can (and in the context of this project, should) be obtained. Students may unwittingly apply defects to the entities they draw which can have a knock-on effect later in the scoring process, particularly with regard to grouping and pitch identification (\cref{sec:pitch-identification}).

We could use island centres? But Notes may not be joined up
We could get over this with region growing until we find an island? but is there a better way?

Yes - Pixel density as in \cref{benoptical}

\todo[inline,color=red]{Pixel density to find centre even if there's a hold}



