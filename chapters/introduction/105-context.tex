\section{Context}

The earliest sources of Western European music recorded on manuscript can be dated back to the late 9th century\footnote{\url{http://en.wikipedia.org/wiki/Music_manuscript}, accessed 13th June 2014} and most likely emerged from the need for a consistent framework by which music could be reproduced and shared, as opposed to just passed on by word of mouth through the generations.

Where once composers could only write and reproduce music by hand, even having to draw their own staff lines, the advent of pre-printed paper allowed rapid composition on a rigid and regular framework. Further advances in technology let to music typesetting and printed music scores which suddenly enabled rapid and faithful reproduction of musical works.

Since then, the advent of personal computers, tablets, scanners and home printers means that with the right software such as Avid's Sibelius\footnote{See http://www.avid.com/US/products/Sibelius/features}, a composer can delegate the task of manuscript representation, notation, layout, and conformity to conventions to a piece of software, while they concentrate on the actual composition.

The benefits technology has brought to the communication of musical notation should not be understated, however in the learning stages of music theory, handwritten notation is a critical developmental tool which takes many students considerable time to master. I believe technology has not yet been used to its full potential in this area, something this project aims to change.

A variety of software and applications (\cref{sec:music-theory-apps}) now exists for learning advanced music theory concepts, composing scores, digitising existing handwritten music and a great many other uses. However in the realm of basic music theory, primarily taught to young children when they start their first instrument, it seems the available toolset is limited to the same exercise books, manuscript paper and repetitive practice that it always has been. Indeed in researching this paper, I discovered that in the 15 years since I first came across music theory myself, the same books and learning styles are still being applied.

Now, this doesn't mean they're in any way \emph{ineffective}, indeed one could quite rightly assume the opposite, these learning styles have demonstrably stood the test of time. In those same 15 years, as noted above, there have been significant advances in the use of technology to support the notation of music in other ways.  However helping students to learn the basics of hand forming musical notation has been neglected.

Now that children are getting more an more used to computers, tablets, games and interactive learning styles, it is my aim to try and bring some of this technology into the area of fundamental music theory and manuscript notation.