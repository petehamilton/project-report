\section{Motivation}

In the national curriculum at the moment in primary education, children aren't taught to write or even read musical notation \Parencite{Attai5} but when they're learning an instrument and it gets to the point where they need to take grades, this is something they need to be able to do \Parencite{whyMusicTheory}.

Learning to properly write music takes time and practice, much as when children are learning normal handwriting skills and so most instrumental teachers would ideally like to teach their children written music theory alongside their normal one-on-one tuition. However there often isn't an abundance of time to spend on this.

The options available to student and teacher are therefore limited to going over the topic during lesson time, or sending a child away with some manuscript paper\footnote{An example of blank manuscript paper can be found at \url{http://www.blanksheetmusic.net/}} and `homework', then marking it when they have their next lesson. Sadly this is often a very ineffective method of learning as without guidance and feedback on a regular basis, it's all too easy for a student to accidentally reinforce bad habits, tighter unsupervised feedback loop is needed.

Whilst several solutions exist which can help children with academic theory, the subject of notation and manuscript has so far not been tackled. Musical OMR\footnote{Converting handwritten music to digital formats} is a similar field but products in that area typically focus on converting draft or rough handwritten notation into a digital format, do not provide feedback are not educationally based.

In short, no solution exists which is capable of helping a child learn to correctly write musical notation whilst providing feedback other than their music teacher.

Since touch-screen and stylus interfaces are now both more accurate and more widespread than they used to be, I propose a solution which will allow a child to practice their notation via the touch-screen as an input, analyse the drawing and score it, not only on it's general accuracy, but also on specific music-oriented aspects such as shape, height and position.

Once the various scores have been established, the child would be given specific and helpful feedback to encourage desired behaviour as talked about in Chapter \ref{section:engagement}.
