\section{Motivation}

In the national curriculum for primary education, children aren't taught to write or even read musical notation \parencite{Attai5} but when they're learning an instrument and it gets to the point where they need to take grades, this is something they must be able to do \parencite{whyMusicTheory}.

Learning to properly write music takes time and practice, much as when children are learning normal handwriting skills so naturally most instrumental teachers would ideally like to teach their children written music theory alongside their normal one-on-one tuition.

However there often isn't an abundance of time to spend on this and so the options available to student and teacher are therefore limited to going over the topic during lesson time, or sending a child away with some manuscript paper\footnote{An example of blank manuscript paper can be found at \url{http://www.blanksheetmusic.net/}} and `homework', then marking it when they have their next lesson. Sadly this is often a very ineffective method of learning as without guidance and feedback on a regular basis, it's all too easy for a student to accidentally reinforce bad habits, a \textbf{faster unsupervised feedback loop} is needed.

Whilst several solutions exist which can help children with academic theory as we can see in \cref{sec:music-theory-apps}, the subject of handling actual notation and manuscript has so far not been tackled. \textbf{\acrfull{OMR}} is a field I draw on heavily in this dissertation field, but commericial and research applications in that area typically focus on converting draft or rough handwritten notation into a digital format and are not educationally based, as such no feedback is provided and often extracting feedback would be hard given the methods used to analyse the scores.

In short, \textbf{no application exists which is capable of helping a child learn to correctly write musical notation whilst simultaneously providing feedback} other than their music teacher. This project aims to make a start towards filling that void.

Since touch-screen and stylus interfaces are now both more accurate and more widespread than they used to be (see \cref{sec:medium-selection} for more details), I propose a solution which will allow a child to practice their notation via the touch-screen as an input, analyse the drawing and score it, not only on it's pictoral\todo{hate this, find a better word} accuracy, but also on more specific music-oriented aspects such as stem length and angle, key signatures and location on the stave\footnote{See \cref{sec:music-theory-stave}}.

Once the various scores have been established, the child would be given specific and helpful feedback to encourage desired behaviour as talked about in Chapter \cref{section:engagement}.
