\title{Title}
\author{Author Name}

%----------------------------------------------------------------------------------------
% USEFUL COMMANDS
%----------------------------------------------------------------------------------------

\newcommand{\ie}{i.\,e.}
\newcommand{\Ie}{I.\,e.}
\newcommand{\eg}{e.\,g.}
\newcommand{\Eg}{E.\,g.}

\newcounter{dummy} % Necessary for correct hyperlinks (to index, bib, etc.)


%----------------------------------------------------------------------------------------
% CODE LISTINGS SETUP
%----------------------------------------------------------------------------------------

\usepackage{listings}
%\lstset{emph={trueIndex,root},emphstyle=\color{BlueViolet}}%\underbar} % for special keywords
\lstset{language=[LaTeX]Tex, % Specify the language for listings here
keywordstyle=\color{RoyalBlue}, % Add \bfseries for bold
basicstyle=\small\ttfamily, % Makes listings a smaller font size and a different font
%identifierstyle=\color{NavyBlue}, % Color of text inside brackets
commentstyle=\color{Green}\ttfamily, % Color of comments
stringstyle=\rmfamily, % Font type to use for strings
numbers=left, % Change left to none to remove line numbers
numberstyle=\scriptsize, % Font size of the line numbers
stepnumber=5, % Increment of line numbers
numbersep=8pt, % Distance of line numbers from code listing
showstringspaces=false, % Sets whether spaces in strings should appear underlined
breaklines=true, % Force the code to stay in the confines of the listing box
%frameround=ftff, % Uncomment for rounded frame
frame=single, % Frame border - none/leftline/topline/bottomline/lines/single/shadowbox/L
belowcaptionskip=.75\baselineskip % Space after the "Listing #: Desciption" text and the listing box
}

%----------------------------------------------------------------------------------------
% FLOATS: TABLES, FIGURES AND CAPTIONS SETUP
%----------------------------------------------------------------------------------------

\usepackage{tabularx} % Better tables
\setlength{\extrarowheight}{3pt} % Increase table row height
\newcommand{\tableheadline}[1]{\multicolumn{1}{c}{\spacedlowsmallcaps{#1}}}
\newcommand{\myfloatalign}{\centering} % To be used with each float for alignment
\usepackage{caption}
\captionsetup{format=hang,font=small}
\usepackage{subfig}
